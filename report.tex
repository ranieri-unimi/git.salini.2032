% Options for packages loaded elsewhere
\PassOptionsToPackage{unicode}{hyperref}
\PassOptionsToPackage{hyphens}{url}
%
\documentclass[
]{article}
\usepackage{amsmath,amssymb}
\usepackage{lmodern}
\usepackage{iftex}
\ifPDFTeX
  \usepackage[T1]{fontenc}
  \usepackage[utf8]{inputenc}
  \usepackage{textcomp} % provide euro and other symbols
\else % if luatex or xetex
  \usepackage{unicode-math}
  \defaultfontfeatures{Scale=MatchLowercase}
  \defaultfontfeatures[\rmfamily]{Ligatures=TeX,Scale=1}
\fi
% Use upquote if available, for straight quotes in verbatim environments
\IfFileExists{upquote.sty}{\usepackage{upquote}}{}
\IfFileExists{microtype.sty}{% use microtype if available
  \usepackage[]{microtype}
  \UseMicrotypeSet[protrusion]{basicmath} % disable protrusion for tt fonts
}{}
\makeatletter
\@ifundefined{KOMAClassName}{% if non-KOMA class
  \IfFileExists{parskip.sty}{%
    \usepackage{parskip}
  }{% else
    \setlength{\parindent}{0pt}
    \setlength{\parskip}{6pt plus 2pt minus 1pt}}
}{% if KOMA class
  \KOMAoptions{parskip=half}}
\makeatother
\usepackage{xcolor}
\usepackage[margin=1in]{geometry}
\usepackage{color}
\usepackage{fancyvrb}
\newcommand{\VerbBar}{|}
\newcommand{\VERB}{\Verb[commandchars=\\\{\}]}
\DefineVerbatimEnvironment{Highlighting}{Verbatim}{commandchars=\\\{\}}
% Add ',fontsize=\small' for more characters per line
\usepackage{framed}
\definecolor{shadecolor}{RGB}{248,248,248}
\newenvironment{Shaded}{\begin{snugshade}}{\end{snugshade}}
\newcommand{\AlertTok}[1]{\textcolor[rgb]{0.94,0.16,0.16}{#1}}
\newcommand{\AnnotationTok}[1]{\textcolor[rgb]{0.56,0.35,0.01}{\textbf{\textit{#1}}}}
\newcommand{\AttributeTok}[1]{\textcolor[rgb]{0.77,0.63,0.00}{#1}}
\newcommand{\BaseNTok}[1]{\textcolor[rgb]{0.00,0.00,0.81}{#1}}
\newcommand{\BuiltInTok}[1]{#1}
\newcommand{\CharTok}[1]{\textcolor[rgb]{0.31,0.60,0.02}{#1}}
\newcommand{\CommentTok}[1]{\textcolor[rgb]{0.56,0.35,0.01}{\textit{#1}}}
\newcommand{\CommentVarTok}[1]{\textcolor[rgb]{0.56,0.35,0.01}{\textbf{\textit{#1}}}}
\newcommand{\ConstantTok}[1]{\textcolor[rgb]{0.00,0.00,0.00}{#1}}
\newcommand{\ControlFlowTok}[1]{\textcolor[rgb]{0.13,0.29,0.53}{\textbf{#1}}}
\newcommand{\DataTypeTok}[1]{\textcolor[rgb]{0.13,0.29,0.53}{#1}}
\newcommand{\DecValTok}[1]{\textcolor[rgb]{0.00,0.00,0.81}{#1}}
\newcommand{\DocumentationTok}[1]{\textcolor[rgb]{0.56,0.35,0.01}{\textbf{\textit{#1}}}}
\newcommand{\ErrorTok}[1]{\textcolor[rgb]{0.64,0.00,0.00}{\textbf{#1}}}
\newcommand{\ExtensionTok}[1]{#1}
\newcommand{\FloatTok}[1]{\textcolor[rgb]{0.00,0.00,0.81}{#1}}
\newcommand{\FunctionTok}[1]{\textcolor[rgb]{0.00,0.00,0.00}{#1}}
\newcommand{\ImportTok}[1]{#1}
\newcommand{\InformationTok}[1]{\textcolor[rgb]{0.56,0.35,0.01}{\textbf{\textit{#1}}}}
\newcommand{\KeywordTok}[1]{\textcolor[rgb]{0.13,0.29,0.53}{\textbf{#1}}}
\newcommand{\NormalTok}[1]{#1}
\newcommand{\OperatorTok}[1]{\textcolor[rgb]{0.81,0.36,0.00}{\textbf{#1}}}
\newcommand{\OtherTok}[1]{\textcolor[rgb]{0.56,0.35,0.01}{#1}}
\newcommand{\PreprocessorTok}[1]{\textcolor[rgb]{0.56,0.35,0.01}{\textit{#1}}}
\newcommand{\RegionMarkerTok}[1]{#1}
\newcommand{\SpecialCharTok}[1]{\textcolor[rgb]{0.00,0.00,0.00}{#1}}
\newcommand{\SpecialStringTok}[1]{\textcolor[rgb]{0.31,0.60,0.02}{#1}}
\newcommand{\StringTok}[1]{\textcolor[rgb]{0.31,0.60,0.02}{#1}}
\newcommand{\VariableTok}[1]{\textcolor[rgb]{0.00,0.00,0.00}{#1}}
\newcommand{\VerbatimStringTok}[1]{\textcolor[rgb]{0.31,0.60,0.02}{#1}}
\newcommand{\WarningTok}[1]{\textcolor[rgb]{0.56,0.35,0.01}{\textbf{\textit{#1}}}}
\usepackage{graphicx}
\makeatletter
\def\maxwidth{\ifdim\Gin@nat@width>\linewidth\linewidth\else\Gin@nat@width\fi}
\def\maxheight{\ifdim\Gin@nat@height>\textheight\textheight\else\Gin@nat@height\fi}
\makeatother
% Scale images if necessary, so that they will not overflow the page
% margins by default, and it is still possible to overwrite the defaults
% using explicit options in \includegraphics[width, height, ...]{}
\setkeys{Gin}{width=\maxwidth,height=\maxheight,keepaspectratio}
% Set default figure placement to htbp
\makeatletter
\def\fps@figure{htbp}
\makeatother
\setlength{\emergencystretch}{3em} % prevent overfull lines
\providecommand{\tightlist}{%
  \setlength{\itemsep}{0pt}\setlength{\parskip}{0pt}}
\setcounter{secnumdepth}{-\maxdimen} % remove section numbering
\ifLuaTeX
  \usepackage{selnolig}  % disable illegal ligatures
\fi
\IfFileExists{bookmark.sty}{\usepackage{bookmark}}{\usepackage{hyperref}}
\IfFileExists{xurl.sty}{\usepackage{xurl}}{} % add URL line breaks if available
\urlstyle{same} % disable monospaced font for URLs
\hypersetup{
  hidelinks,
  pdfcreator={LaTeX via pandoc}}

\author{}
\date{\vspace{-2.5em}}

\begin{document}

\hypertarget{laddering-unite}{%
\section{Laddering Unite}\label{laddering-unite}}

\begin{Shaded}
\begin{Highlighting}[]
\FunctionTok{library}\NormalTok{(scales)}
\end{Highlighting}
\end{Shaded}

Pokémon Unite is an online video game. Matches consist in 10 minutes
where two teams, each with 5 players, must archive the highest score to
win. The player will partake in team-based battles: two teams of five
Pokémons battle against each other.

At the beginning of a match, players will be able to select which
Pokémon they want to play. Pokémons have different base stats and are
better in some aspects of the games than others. For example, someone
have high defence which allows them to play as a shield to Pokémons
which have low health points, but high attack.

While battling, the player is also able to level up once they defeat
enough Pokémon. Once a Pokémon is defeated, opponent or wild, the player
collects Balls which can then be dispensed to the opposing team's goal
zones, and therefore collect points for the team. The player can heal by
standing in their own team's goal zone or scoring points. The winner is
decided by the higher final score of both teams once the battle timer is
over.

\hypertarget{datasets}{%
\subsection{Datasets}\label{datasets}}

Two tike of data source are used.

\hypertarget{pokuxe9mon-features}{%
\subsubsection{Pokémon Features}\label{pokuxe9mon-features}}

The first dataset carries the features of each Pokémon. Raw data are
stored by Pokémon level, let's merge them:

\hypertarget{data-ingestion}{%
\paragraph{Data Ingestion}\label{data-ingestion}}

\begin{Shaded}
\begin{Highlighting}[]
\NormalTok{read\_stats }\OtherTok{=} \ControlFlowTok{function}\NormalTok{()\{}
\NormalTok{  df }\OtherTok{=} \FunctionTok{data.frame}\NormalTok{()}
  \ControlFlowTok{for}\NormalTok{(i }\ControlFlowTok{in} \DecValTok{1}\SpecialCharTok{:}\DecValTok{15}\NormalTok{)\{}
\NormalTok{    uri }\OtherTok{=} \FunctionTok{sprintf}\NormalTok{(}\StringTok{"datasets/stats/\%s.csv"}\NormalTok{, i)}
\NormalTok{    dfi }\OtherTok{=} \FunctionTok{read.csv}\NormalTok{(uri)}
\NormalTok{    dfi}\SpecialCharTok{$}\NormalTok{level }\OtherTok{=}\NormalTok{ i}
\NormalTok{    df }\OtherTok{=} \FunctionTok{rbind}\NormalTok{(df, dfi)}
\NormalTok{  \}}
  \FunctionTok{return}\NormalTok{(df)}
\NormalTok{\}}
\end{Highlighting}
\end{Shaded}

\begin{Shaded}
\begin{Highlighting}[]
\NormalTok{df }\OtherTok{=} \FunctionTok{read\_stats}\NormalTok{()}
\end{Highlighting}
\end{Shaded}

Names are added for visualization purpose:

\begin{Shaded}
\begin{Highlighting}[]
\NormalTok{df\_rn }\OtherTok{=} \FunctionTok{paste}\NormalTok{(}
  \FunctionTok{toupper}\NormalTok{(}\FunctionTok{substr}\NormalTok{(df}\SpecialCharTok{$}\NormalTok{name, }\DecValTok{1}\NormalTok{,}\DecValTok{4}\NormalTok{)),}
\NormalTok{  df}\SpecialCharTok{$}\NormalTok{level,}
  \AttributeTok{sep=}\StringTok{""}\NormalTok{)}

\FunctionTok{row.names}\NormalTok{(df) }\OtherTok{=}\NormalTok{ df\_rn}
\NormalTok{df}\SpecialCharTok{$}\NormalTok{name }\OtherTok{=} \ConstantTok{NULL}
\end{Highlighting}
\end{Shaded}

\begin{Shaded}
\begin{Highlighting}[]
\NormalTok{df}\SpecialCharTok{$}\NormalTok{speed }\OtherTok{=} \ConstantTok{NULL}
\end{Highlighting}
\end{Shaded}

Each Pokémon is described by 8 features:

\begin{itemize}
\tightlist
\item
  Health Points (HP)
\item
  Attack
\item
  Defense
\item
  Special Attack
\item
  Special Defense
\item
  Critical Rate
\item
  Cooldown Reduction Percentage
\item
  Life Steal Percentage
\end{itemize}

All of them are positive scalar, let's normalize them:

\begin{Shaded}
\begin{Highlighting}[]
\ControlFlowTok{for}\NormalTok{(i }\ControlFlowTok{in} \DecValTok{1}\SpecialCharTok{:}\FunctionTok{length}\NormalTok{(df))\{}
\NormalTok{  df[i] }\OtherTok{=}\NormalTok{ ((df[i] }\SpecialCharTok{{-}} \FunctionTok{min}\NormalTok{(df[i])) }\SpecialCharTok{/}\NormalTok{(}\FunctionTok{max}\NormalTok{(df[i])}\SpecialCharTok{{-}}\FunctionTok{min}\NormalTok{(df[i])))}
\NormalTok{\}}

\NormalTok{df }\OtherTok{=}\NormalTok{ df[df}\SpecialCharTok{$}\NormalTok{level }\SpecialCharTok{\textgreater{}}\NormalTok{ (}\DecValTok{13}\SpecialCharTok{/}\DecValTok{15}\NormalTok{),]}
\NormalTok{df}\SpecialCharTok{$}\NormalTok{level }\OtherTok{=} \ConstantTok{NULL}
\end{Highlighting}
\end{Shaded}

\hypertarget{dimensionality-reduction}{%
\paragraph{Dimensionality Reduction}\label{dimensionality-reduction}}

The features of Pokémons are a bit redundant, let's reduce dimensions
with PCA:

\begin{Shaded}
\begin{Highlighting}[]
\NormalTok{res }\OtherTok{=} \FunctionTok{princomp}\NormalTok{(df, }\AttributeTok{cor=}\NormalTok{T)}
\FunctionTok{summary}\NormalTok{(res)}
\end{Highlighting}
\end{Shaded}

\begin{verbatim}
## Importance of components:
##                           Comp.1    Comp.2    Comp.3     Comp.4     Comp.5
## Standard deviation     1.9447788 1.6564725 0.9145924 0.48579637 0.40637021
## Proportion of Variance 0.4727706 0.3429877 0.1045599 0.02949976 0.02064209
## Cumulative Proportion  0.4727706 0.8157582 0.9203182 0.94981792 0.97046001
##                            Comp.6     Comp.7      Comp.8
## Standard deviation     0.34521247 0.29903741 0.166507853
## Proportion of Variance 0.01489646 0.01117792 0.003465608
## Cumulative Proportion  0.98535647 0.99653439 1.000000000
\end{verbatim}

\begin{Shaded}
\begin{Highlighting}[]
\FunctionTok{screeplot}\NormalTok{(res)}
\end{Highlighting}
\end{Shaded}

\includegraphics{report_files/figure-latex/unnamed-chunk-8-1.pdf}

\begin{Shaded}
\begin{Highlighting}[]
\FunctionTok{plot}\NormalTok{(res}\SpecialCharTok{$}\NormalTok{scores, }\AttributeTok{cex=}\FloatTok{0.0}\NormalTok{)}
\FunctionTok{text}\NormalTok{(res}\SpecialCharTok{$}\NormalTok{scores, }\FunctionTok{rownames}\NormalTok{(df), }\AttributeTok{cex=}\FloatTok{0.6}\NormalTok{)}
\FunctionTok{abline}\NormalTok{(}\AttributeTok{h=}\DecValTok{0}\NormalTok{, }\AttributeTok{v=}\DecValTok{0}\NormalTok{)}
\end{Highlighting}
\end{Shaded}

\includegraphics{report_files/figure-latex/unnamed-chunk-8-2.pdf}

\begin{Shaded}
\begin{Highlighting}[]
\NormalTok{pr.var}\OtherTok{=}\NormalTok{res}\SpecialCharTok{$}\NormalTok{sdev}\SpecialCharTok{\^{}}\DecValTok{2}
\NormalTok{pve}\OtherTok{=}\NormalTok{pr.var}\SpecialCharTok{/}\FunctionTok{sum}\NormalTok{(pr.var)}

\FunctionTok{plot}\NormalTok{(}\FunctionTok{cumsum}\NormalTok{(pve), }\AttributeTok{xlab=}\StringTok{"Principal Component"}\NormalTok{, }\AttributeTok{ylab=}\StringTok{"Cumulative Proportion of Variance Explained"}\NormalTok{, }\AttributeTok{ylim=}\FunctionTok{c}\NormalTok{(}\DecValTok{0}\NormalTok{,}\DecValTok{1}\NormalTok{),}\AttributeTok{type=}\StringTok{\textquotesingle{}b\textquotesingle{}}\NormalTok{)}
\end{Highlighting}
\end{Shaded}

\includegraphics{report_files/figure-latex/unnamed-chunk-9-1.pdf}

The first two principal components are enough to explain the 80\% of the
variability among the Pokemon stats. Let's find out their composition:

\begin{Shaded}
\begin{Highlighting}[]
\NormalTok{res}\SpecialCharTok{$}\NormalTok{loading[,}\DecValTok{1}\SpecialCharTok{:}\DecValTok{2}\NormalTok{]}
\end{Highlighting}
\end{Shaded}

\begin{verbatim}
##                    Comp.1      Comp.2
## hp              0.1476821  0.55601887
## attack          0.2797591 -0.25474867
## defense         0.3728057  0.38087074
## sp_attack      -0.4583060  0.04224364
## sp_defense      0.3311075  0.44716195
## crit_rate       0.3259758 -0.40775085
## cooldown_reduc -0.4495800  0.06297616
## life_steal      0.3645296 -0.33002806
\end{verbatim}

It seems that the first component measure the adverseness to be a core
Special Attacker: very low \texttt{sp\_attack} and
\texttt{cooldown\_reduc}. The second component seems to measure the
defensiveness: low \texttt{attack}, \texttt{crit\_rate} and
\texttt{life\_stea}l, against high \texttt{defense},
\texttt{sp\_defence} and \texttt{hp}.

\hypertarget{clustering}{%
\paragraph{Clustering}\label{clustering}}

However also the number of possible playable Pokemon is quite high. K
means is applied to delineate groups of interchangeable Pokémon. K means
cannot be applied over the original point cloud because features are not
comparable (distances between points are not euclidean, and an
appropriate distance has not been found)

\begin{Shaded}
\begin{Highlighting}[]
\FunctionTok{biplot}\NormalTok{(res, }\AttributeTok{cex=}\FloatTok{0.5}\NormalTok{)}
\FunctionTok{abline}\NormalTok{(}\AttributeTok{h=}\DecValTok{0}\NormalTok{, }\AttributeTok{v=}\DecValTok{0}\NormalTok{)}
\end{Highlighting}
\end{Shaded}

\includegraphics{report_files/figure-latex/unnamed-chunk-11-1.pdf}

\begin{Shaded}
\begin{Highlighting}[]
\NormalTok{wssplot }\OtherTok{\textless{}{-}} \ControlFlowTok{function}\NormalTok{(data, }\AttributeTok{nc=}\DecValTok{15}\NormalTok{, }\AttributeTok{seed=}\DecValTok{1234}\NormalTok{)\{}
\NormalTok{  wss }\OtherTok{=}\NormalTok{ (}\FunctionTok{nrow}\NormalTok{(data)}\SpecialCharTok{{-}}\DecValTok{1}\NormalTok{)}\SpecialCharTok{*}\FunctionTok{sum}\NormalTok{(}\FunctionTok{apply}\NormalTok{(data,}\DecValTok{2}\NormalTok{,var))}
  \ControlFlowTok{for}\NormalTok{ (i }\ControlFlowTok{in} \DecValTok{2}\SpecialCharTok{:}\NormalTok{nc)\{}
    \FunctionTok{set.seed}\NormalTok{(seed)}
\NormalTok{    wss[i] }\OtherTok{\textless{}{-}} \FunctionTok{sum}\NormalTok{(}\FunctionTok{kmeans}\NormalTok{(data, }\AttributeTok{centers=}\NormalTok{i)}\SpecialCharTok{$}\NormalTok{withinss)\}}
  \FunctionTok{plot}\NormalTok{(}\DecValTok{1}\SpecialCharTok{:}\NormalTok{nc, wss, }\AttributeTok{type=}\StringTok{"b"}\NormalTok{, }\AttributeTok{xlab=}\StringTok{"Number of Clusters"}\NormalTok{,}
       \AttributeTok{ylab=}\StringTok{"Within groups sum of squares"}\NormalTok{)\}}

\FunctionTok{wssplot}\NormalTok{(df[}\DecValTok{2}\SpecialCharTok{:}\FunctionTok{length}\NormalTok{(df)], }\AttributeTok{nc=}\DecValTok{12}\NormalTok{) }
\end{Highlighting}
\end{Shaded}

\includegraphics{report_files/figure-latex/unnamed-chunk-12-1.pdf}

\begin{Shaded}
\begin{Highlighting}[]
\FunctionTok{library}\NormalTok{(cluster)}

\FunctionTok{clusplot}\NormalTok{(df, }\FunctionTok{kmeans}\NormalTok{(df, }\AttributeTok{centers=}\DecValTok{6}\NormalTok{)}\SpecialCharTok{$}\NormalTok{cluster, }
         \AttributeTok{main=}\StringTok{\textquotesingle{}2D representation of the Cluster solution\textquotesingle{}}\NormalTok{,}
         \AttributeTok{color=}\ConstantTok{TRUE}\NormalTok{, }\AttributeTok{shade=}\ConstantTok{TRUE}\NormalTok{,}
         \AttributeTok{labels=}\DecValTok{2}\NormalTok{, }\AttributeTok{lines=}\DecValTok{0}\NormalTok{, }\AttributeTok{cex=}\FloatTok{0.5}\NormalTok{)}
\end{Highlighting}
\end{Shaded}

\includegraphics{report_files/figure-latex/unnamed-chunk-13-1.pdf}

We'll cheat a bit: a better clustering is found using \textbf{DB SCAN}

\begin{Shaded}
\begin{Highlighting}[]
\CommentTok{\# Compute DBSCAN using fpc package}
\FunctionTok{library}\NormalTok{(}\StringTok{"fpc"}\NormalTok{)}
\end{Highlighting}
\end{Shaded}

\begin{verbatim}
## Warning: package 'fpc' was built under R version 4.1.3
\end{verbatim}

\begin{Shaded}
\begin{Highlighting}[]
\NormalTok{db }\OtherTok{=}\NormalTok{ fpc}\SpecialCharTok{::}\FunctionTok{dbscan}\NormalTok{(res}\SpecialCharTok{$}\NormalTok{scores, }\AttributeTok{eps =} \FloatTok{1.6}\NormalTok{, }\AttributeTok{MinPts =} \DecValTok{2}\NormalTok{)}

\CommentTok{\# Plot DBSCAN results}
\FunctionTok{library}\NormalTok{(}\StringTok{"factoextra"}\NormalTok{)}
\end{Highlighting}
\end{Shaded}

\begin{verbatim}
## Warning: package 'factoextra' was built under R version 4.1.3
\end{verbatim}

\begin{verbatim}
## Loading required package: ggplot2
\end{verbatim}

\begin{verbatim}
## Warning: package 'ggplot2' was built under R version 4.1.3
\end{verbatim}

\begin{verbatim}
## Welcome! Want to learn more? See two factoextra-related books at https://goo.gl/ve3WBa
\end{verbatim}

\begin{Shaded}
\begin{Highlighting}[]
\FunctionTok{fviz\_cluster}\NormalTok{(db, }\AttributeTok{data =}\NormalTok{ res}\SpecialCharTok{$}\NormalTok{scores, }\AttributeTok{stand =} \ConstantTok{FALSE}\NormalTok{,}
\AttributeTok{ellipse =} \ConstantTok{TRUE}\NormalTok{, }\AttributeTok{show.clust.cent =} \ConstantTok{FALSE}\NormalTok{,}
\AttributeTok{geom =} \StringTok{"point"}\NormalTok{,}\AttributeTok{palette =} \StringTok{"jco"}\NormalTok{, }\AttributeTok{ggtheme =} \FunctionTok{theme\_classic}\NormalTok{())}
\end{Highlighting}
\end{Shaded}

\includegraphics{report_files/figure-latex/unnamed-chunk-14-1.pdf}

\begin{Shaded}
\begin{Highlighting}[]
\CommentTok{\# clookup = data.frame()}
\CommentTok{\# }
\CommentTok{\# for(c in 1:length(df))\{}
\CommentTok{\#   clookup = rbind(clookup, c(c, row.names(df)[c], db$cluster[c]))}
\CommentTok{\# \}}
\CommentTok{\# }
\CommentTok{\# colnames(clookup) = c(\textquotesingle{}id\textquotesingle{},\textquotesingle{}name\textquotesingle{}, \textquotesingle{}cluster\textquotesingle{})}
\end{Highlighting}
\end{Shaded}

\begin{Shaded}
\begin{Highlighting}[]
\NormalTok{clust\_map }\OtherTok{=} \FunctionTok{read.csv}\NormalTok{(}\StringTok{"datasets/clusters.csv"}\NormalTok{)}
\end{Highlighting}
\end{Shaded}

\hypertarget{match-results}{%
\subsubsection{Match Results}\label{match-results}}

\begin{Shaded}
\begin{Highlighting}[]
\NormalTok{dt }\OtherTok{=} \FunctionTok{read.csv}\NormalTok{(}\StringTok{\textquotesingle{}datasets/matches.csv\textquotesingle{}}\NormalTok{)}
\end{Highlighting}
\end{Shaded}

The second dataset carries the performances for 50 matches of all team
players. Relevant metrics measured are:

\begin{itemize}
\tightlist
\item
  Level reached by the player
\item
  Individual points scored
\item
  Nº of kills
\item
  Nº of assist
\item
  Nº of interrupt
\item
  HP damage done
\item
  HP damage taken
\item
  HP self-cured
\end{itemize}

\begin{Shaded}
\begin{Highlighting}[]
\NormalTok{pkmn\_count }\OtherTok{=} \FunctionTok{aggregate}\NormalTok{(dt}\SpecialCharTok{$}\NormalTok{score, }\AttributeTok{by=}\FunctionTok{list}\NormalTok{(}\AttributeTok{pokemon=}\NormalTok{dt}\SpecialCharTok{$}\NormalTok{pokemon), }\AttributeTok{FUN=}\NormalTok{length)}
\NormalTok{pkmn\_count[}\FunctionTok{order}\NormalTok{(}\SpecialCharTok{{-}}\NormalTok{pkmn\_count}\SpecialCharTok{$}\NormalTok{x),][}\DecValTok{1}\SpecialCharTok{:}\DecValTok{7}\NormalTok{,]}
\end{Highlighting}
\end{Shaded}

\begin{verbatim}
##      pokemon  x
## 10   Crustle 44
## 29    MrMime 35
## 7  Charizard 28
## 31   Pikachu 24
## 28       Mew 23
## 17    Espeon 19
## 20    Gengar 19
\end{verbatim}

Players do not choose Pokémon uniformly, somehow cluster delineated
above will help to analyze the whole picture of a Pokémon role-play,
instead of focusing on each single Pokémon.

\begin{Shaded}
\begin{Highlighting}[]
\FunctionTok{shapiro.test}\NormalTok{(dt}\SpecialCharTok{$}\NormalTok{score)}
\end{Highlighting}
\end{Shaded}

\begin{verbatim}
## 
##  Shapiro-Wilk normality test
## 
## data:  dt$score
## W = 0.91439, p-value = 3.301e-16
\end{verbatim}

\begin{Shaded}
\begin{Highlighting}[]
\FunctionTok{library}\NormalTok{(ggpubr)}
\end{Highlighting}
\end{Shaded}

\begin{verbatim}
## Warning: package 'ggpubr' was built under R version 4.1.3
\end{verbatim}

\begin{Shaded}
\begin{Highlighting}[]
\FunctionTok{ggqqplot}\NormalTok{(dt}\SpecialCharTok{$}\NormalTok{score)}
\end{Highlighting}
\end{Shaded}

\includegraphics{report_files/figure-latex/unnamed-chunk-20-1.pdf}

\begin{Shaded}
\begin{Highlighting}[]
\FunctionTok{ggdensity}\NormalTok{(dt, }\AttributeTok{x =} \StringTok{"score"}\NormalTok{, }\AttributeTok{fill =} \StringTok{"lightgray"}\NormalTok{, }\AttributeTok{title =} \StringTok{"score"}\NormalTok{) }\SpecialCharTok{+}
  \FunctionTok{stat\_overlay\_normal\_density}\NormalTok{(}\AttributeTok{color =} \StringTok{"red"}\NormalTok{, }\AttributeTok{linetype =} \StringTok{"dashed"}\NormalTok{)}
\end{Highlighting}
\end{Shaded}

\includegraphics{report_files/figure-latex/unnamed-chunk-20-2.pdf}

The score archived by each single player is not a reliable dependent
variable, it's weirdly bi-modal due to game mechanics, and somewhat
poissionian. Instead, the pretty binary win-lose outcome will be
considered.

\hypertarget{optimize-gameplay---forward-selection-and-logistic-regression}{%
\subsection{Optimize Gameplay - Forward Selection and Logistic
Regression}\label{optimize-gameplay---forward-selection-and-logistic-regression}}

Let's see regression for two different Pokémons. Forward Selection is
implemented to point which match stats are more relevant for each
Pokémon.

\begin{Shaded}
\begin{Highlighting}[]
\FunctionTok{library}\NormalTok{(leaps)}
\end{Highlighting}
\end{Shaded}

\begin{verbatim}
## Warning: package 'leaps' was built under R version 4.1.3
\end{verbatim}

\begin{Shaded}
\begin{Highlighting}[]
\NormalTok{cols }\OtherTok{=} \FunctionTok{c}\NormalTok{(}\StringTok{\textquotesingle{}level\textquotesingle{}}\NormalTok{, }\StringTok{\textquotesingle{}score\textquotesingle{}}\NormalTok{, }\StringTok{\textquotesingle{}kill\textquotesingle{}}\NormalTok{, }\StringTok{\textquotesingle{}assist\textquotesingle{}}\NormalTok{, }\StringTok{\textquotesingle{}interrupt\textquotesingle{}}\NormalTok{, }\StringTok{\textquotesingle{}damage\_done\textquotesingle{}}\NormalTok{, }\StringTok{\textquotesingle{}damage\_taken\textquotesingle{}}\NormalTok{, }\StringTok{\textquotesingle{}damage\_healed\textquotesingle{}}\NormalTok{)}
\end{Highlighting}
\end{Shaded}

\begin{quote}
\emph{Why Y and Log X? Interpretation purpose:}

\emph{\textbf{LINEAR}: A 1\% increase in X would lead to a β\%
increase/decrease in Y}

\emph{\textbf{LOGIT}: A k-factor increase in X would lead to a k**β
increase in odds.}

\href{https://stats.stackexchange.com/questions/8318/interpretation-of-log-transformed-predictors-in-logistic-regression}{\emph{https://stats.stackexchange.com/questions/8318/interpretation-of-log-transformed-predictors-in-logistic-regression}}
\end{quote}

\hypertarget{crustle}{%
\subsubsection{Crustle}\label{crustle}}

\begin{Shaded}
\begin{Highlighting}[]
\NormalTok{who }\OtherTok{=} \StringTok{"Crustle"}

\NormalTok{dtw }\OtherTok{=}\NormalTok{ dt[dt}\SpecialCharTok{$}\NormalTok{pokemon }\SpecialCharTok{==}\NormalTok{ who, ]}
\NormalTok{dtw[,cols] }\OtherTok{=} \FunctionTok{log}\NormalTok{(dtw[,cols]}\SpecialCharTok{+}\DecValTok{1}\NormalTok{)}

\NormalTok{regfit.fwd }\OtherTok{=} \FunctionTok{regsubsets}\NormalTok{(win }\SpecialCharTok{\textasciitilde{}}\NormalTok{ level }\SpecialCharTok{+}\NormalTok{ score }\SpecialCharTok{+}\NormalTok{ kill }\SpecialCharTok{+}\NormalTok{ assist }\SpecialCharTok{+}\NormalTok{ interrupt }\SpecialCharTok{+}\NormalTok{ damage\_done }\SpecialCharTok{+}\NormalTok{ damage\_taken }\SpecialCharTok{+}\NormalTok{ damage\_healed, }\AttributeTok{data=}\NormalTok{dtw, }\AttributeTok{method=}\StringTok{"forward"}\NormalTok{)}

\FunctionTok{summary}\NormalTok{(regfit.fwd)}
\end{Highlighting}
\end{Shaded}

\begin{verbatim}
## Subset selection object
## Call: regsubsets.formula(win ~ level + score + kill + assist + interrupt + 
##     damage_done + damage_taken + damage_healed, data = dtw, method = "forward")
## 8 Variables  (and intercept)
##               Forced in Forced out
## level             FALSE      FALSE
## score             FALSE      FALSE
## kill              FALSE      FALSE
## assist            FALSE      FALSE
## interrupt         FALSE      FALSE
## damage_done       FALSE      FALSE
## damage_taken      FALSE      FALSE
## damage_healed     FALSE      FALSE
## 1 subsets of each size up to 8
## Selection Algorithm: forward
##          level score kill assist interrupt damage_done damage_taken
## 1  ( 1 ) " "   " "   " "  "*"    " "       " "         " "         
## 2  ( 1 ) "*"   " "   " "  "*"    " "       " "         " "         
## 3  ( 1 ) "*"   " "   " "  "*"    " "       " "         " "         
## 4  ( 1 ) "*"   " "   " "  "*"    " "       " "         "*"         
## 5  ( 1 ) "*"   " "   "*"  "*"    " "       " "         "*"         
## 6  ( 1 ) "*"   " "   "*"  "*"    " "       "*"         "*"         
## 7  ( 1 ) "*"   "*"   "*"  "*"    " "       "*"         "*"         
## 8  ( 1 ) "*"   "*"   "*"  "*"    "*"       "*"         "*"         
##          damage_healed
## 1  ( 1 ) " "          
## 2  ( 1 ) " "          
## 3  ( 1 ) "*"          
## 4  ( 1 ) "*"          
## 5  ( 1 ) "*"          
## 6  ( 1 ) "*"          
## 7  ( 1 ) "*"          
## 8  ( 1 ) "*"
\end{verbatim}

\begin{Shaded}
\begin{Highlighting}[]
\NormalTok{glm.fit }\OtherTok{\textless{}{-}} \FunctionTok{glm}\NormalTok{(win }\SpecialCharTok{\textasciitilde{}}\NormalTok{ assist }\SpecialCharTok{+}\NormalTok{ level , }\AttributeTok{data =}\NormalTok{ dtw, }\AttributeTok{family=}\FunctionTok{binomial}\NormalTok{(}\AttributeTok{link=}\StringTok{\textquotesingle{}logit\textquotesingle{}}\NormalTok{))}
\FunctionTok{summary}\NormalTok{(glm.fit)}
\end{Highlighting}
\end{Shaded}

\begin{verbatim}
## 
## Call:
## glm(formula = win ~ assist + level, family = binomial(link = "logit"), 
##     data = dtw)
## 
## Deviance Residuals: 
##     Min       1Q   Median       3Q      Max  
## -1.7461  -0.8556   0.2972   0.8821   1.7616  
## 
## Coefficients:
##             Estimate Std. Error z value Pr(>|z|)  
## (Intercept)  -34.787     13.786  -2.523   0.0116 *
## assist         2.939      1.176   2.498   0.0125 *
## level         11.206      4.974   2.253   0.0243 *
## ---
## Signif. codes:  0 '***' 0.001 '**' 0.01 '*' 0.05 '.' 0.1 ' ' 1
## 
## (Dispersion parameter for binomial family taken to be 1)
## 
##     Null deviance: 59.534  on 43  degrees of freedom
## Residual deviance: 45.151  on 41  degrees of freedom
## AIC: 51.151
## 
## Number of Fisher Scoring iterations: 5
\end{verbatim}

Crustle, in order to raise its win odds, have, first of all, to keep its
level high and secondly to assist its teammates during fights.

\hypertarget{pikachu}{%
\subsubsection{Pikachu}\label{pikachu}}

\begin{Shaded}
\begin{Highlighting}[]
\NormalTok{who }\OtherTok{=} \StringTok{"Pikachu"}

\NormalTok{dtw }\OtherTok{=}\NormalTok{ dt[dt}\SpecialCharTok{$}\NormalTok{pokemon }\SpecialCharTok{==}\NormalTok{ who, ]}
\NormalTok{dtw[,cols] }\OtherTok{=} \FunctionTok{log}\NormalTok{(dtw[,cols]}\SpecialCharTok{+}\DecValTok{1}\NormalTok{)}

\NormalTok{regfit.fwd }\OtherTok{=} \FunctionTok{regsubsets}\NormalTok{(win }\SpecialCharTok{\textasciitilde{}}\NormalTok{ level }\SpecialCharTok{+}\NormalTok{ score }\SpecialCharTok{+}\NormalTok{ kill }\SpecialCharTok{+}\NormalTok{ assist }\SpecialCharTok{+}\NormalTok{ interrupt }\SpecialCharTok{+}\NormalTok{ damage\_done }\SpecialCharTok{+}\NormalTok{ damage\_taken }\SpecialCharTok{+}\NormalTok{ damage\_healed, }\AttributeTok{data=}\NormalTok{dtw, }\AttributeTok{method=}\StringTok{"forward"}\NormalTok{)}

\FunctionTok{summary}\NormalTok{(regfit.fwd)}
\end{Highlighting}
\end{Shaded}

\begin{verbatim}
## Subset selection object
## Call: regsubsets.formula(win ~ level + score + kill + assist + interrupt + 
##     damage_done + damage_taken + damage_healed, data = dtw, method = "forward")
## 8 Variables  (and intercept)
##               Forced in Forced out
## level             FALSE      FALSE
## score             FALSE      FALSE
## kill              FALSE      FALSE
## assist            FALSE      FALSE
## interrupt         FALSE      FALSE
## damage_done       FALSE      FALSE
## damage_taken      FALSE      FALSE
## damage_healed     FALSE      FALSE
## 1 subsets of each size up to 8
## Selection Algorithm: forward
##          level score kill assist interrupt damage_done damage_taken
## 1  ( 1 ) " "   " "   " "  " "    " "       " "         "*"         
## 2  ( 1 ) " "   "*"   " "  " "    " "       " "         "*"         
## 3  ( 1 ) " "   "*"   " "  "*"    " "       " "         "*"         
## 4  ( 1 ) " "   "*"   " "  "*"    "*"       " "         "*"         
## 5  ( 1 ) "*"   "*"   " "  "*"    "*"       " "         "*"         
## 6  ( 1 ) "*"   "*"   " "  "*"    "*"       " "         "*"         
## 7  ( 1 ) "*"   "*"   "*"  "*"    "*"       " "         "*"         
## 8  ( 1 ) "*"   "*"   "*"  "*"    "*"       "*"         "*"         
##          damage_healed
## 1  ( 1 ) " "          
## 2  ( 1 ) " "          
## 3  ( 1 ) " "          
## 4  ( 1 ) " "          
## 5  ( 1 ) " "          
## 6  ( 1 ) "*"          
## 7  ( 1 ) "*"          
## 8  ( 1 ) "*"
\end{verbatim}

\begin{Shaded}
\begin{Highlighting}[]
\NormalTok{glm.fit }\OtherTok{\textless{}{-}} \FunctionTok{glm}\NormalTok{(win }\SpecialCharTok{\textasciitilde{}}\NormalTok{ damage\_taken }\SpecialCharTok{+}\NormalTok{ score , }\AttributeTok{data =}\NormalTok{ dtw, }\AttributeTok{family=}\FunctionTok{binomial}\NormalTok{(}\AttributeTok{link=}\StringTok{\textquotesingle{}logit\textquotesingle{}}\NormalTok{))}
\FunctionTok{summary}\NormalTok{(glm.fit)}
\end{Highlighting}
\end{Shaded}

\begin{verbatim}
## 
## Call:
## glm(formula = win ~ damage_taken + score, family = binomial(link = "logit"), 
##     data = dtw)
## 
## Deviance Residuals: 
##      Min        1Q    Median        3Q       Max  
## -1.95316  -0.65115  -0.09791   0.87001   1.51511  
## 
## Coefficients:
##              Estimate Std. Error z value Pr(>|z|)  
## (Intercept)   89.4655    48.5409   1.843   0.0653 .
## damage_taken  -8.7857     4.6469  -1.891   0.0587 .
## score          1.0993     0.5533   1.987   0.0469 *
## ---
## Signif. codes:  0 '***' 0.001 '**' 0.01 '*' 0.05 '.' 0.1 ' ' 1
## 
## (Dispersion parameter for binomial family taken to be 1)
## 
##     Null deviance: 33.104  on 23  degrees of freedom
## Residual deviance: 20.938  on 21  degrees of freedom
## AIC: 26.938
## 
## Number of Fisher Scoring iterations: 6
\end{verbatim}

On the other hand, Pikachu have to avoid damage from opponents, and
prioritize scoring.

\hypertarget{team-synergy}{%
\subsection{Team synergy}\label{team-synergy}}

Winning it's not just a matter of single players behavior, but also of
compatibility between Pokemons. Most of the time a team with balanced
roles and stats is the main key for the victory.

\begin{Shaded}
\begin{Highlighting}[]
\NormalTok{dp }\OtherTok{=} \FunctionTok{read.csv}\NormalTok{(}\StringTok{"datasets/pivot.group.matches.csv"}\NormalTok{)}
\end{Highlighting}
\end{Shaded}

Match results have been pivoted, the metrics that have been kept for
each team are: - The average performance achieved - The count of Pokemon
that were played for each role-cluster

Let's move a bit out of linearity:

\begin{Shaded}
\begin{Highlighting}[]
\CommentTok{\# for( c in names(dp)[55:60])\{}
\CommentTok{\#   dp[paste0(c,\textquotesingle{}\_2\textquotesingle{})] = as.integer(dp[c] \textgreater{} 1)}
\CommentTok{\#   dp[c] = as.integer(dp[c] == 1)}
\CommentTok{\# \}}

\ControlFlowTok{for}\NormalTok{( c }\ControlFlowTok{in} \FunctionTok{names}\NormalTok{(dp)[}\DecValTok{55}\SpecialCharTok{:}\DecValTok{60}\NormalTok{])\{}
\NormalTok{  dp[}\FunctionTok{paste0}\NormalTok{(c,}\StringTok{\textquotesingle{}\_2\textquotesingle{}}\NormalTok{)] }\OtherTok{=}\NormalTok{ (dp[c]}\SpecialCharTok{\^{}}\DecValTok{2}\NormalTok{)[}\DecValTok{1}\NormalTok{]}
\NormalTok{\}}
\end{Highlighting}
\end{Shaded}

\hypertarget{a-balanced-team---logistic-regression}{%
\subsubsection{A Balanced Team - Logistic
Regression}\label{a-balanced-team---logistic-regression}}

To model how team roles should be balanced a logistic regression is
carried out over team compositions, using the role-groups delineated in
the previous clustering. For visualization purpose, label names have
been assigned manually:

\begin{Shaded}
\begin{Highlighting}[]
\FunctionTok{names}\NormalTok{(dp)[}\DecValTok{55}\SpecialCharTok{:}\DecValTok{60}\NormalTok{]}
\end{Highlighting}
\end{Shaded}

\begin{verbatim}
## [1] "support"    "versatile"  "atk_ranged" "sp_atk"     "speedster" 
## [6] "defence"
\end{verbatim}

\begin{Shaded}
\begin{Highlighting}[]
\NormalTok{glm.fit }\OtherTok{\textless{}{-}} \FunctionTok{glm}\NormalTok{(win }\SpecialCharTok{\textasciitilde{}}\NormalTok{ support }\SpecialCharTok{+}\NormalTok{ versatile }\SpecialCharTok{+}\NormalTok{ atk\_ranged }\SpecialCharTok{+}\NormalTok{ sp\_atk }\SpecialCharTok{+}\NormalTok{ speedster }\SpecialCharTok{+}\NormalTok{ defence}
                \SpecialCharTok{+}\NormalTok{ support\_2 }\SpecialCharTok{+}\NormalTok{ versatile\_2 }\SpecialCharTok{+}\NormalTok{ atk\_ranged\_2 }\SpecialCharTok{+}\NormalTok{ sp\_atk\_2 }\SpecialCharTok{+}\NormalTok{ speedster\_2 }\SpecialCharTok{+}\NormalTok{ defence\_2}
\NormalTok{               , }\AttributeTok{data =}\NormalTok{ dp, }\AttributeTok{family=}\FunctionTok{binomial}\NormalTok{(}\AttributeTok{link=}\StringTok{\textquotesingle{}logit\textquotesingle{}}\NormalTok{))}
\FunctionTok{summary}\NormalTok{(glm.fit)}
\end{Highlighting}
\end{Shaded}

\begin{verbatim}
## 
## Call:
## glm(formula = win ~ support + versatile + atk_ranged + sp_atk + 
##     speedster + defence + support_2 + versatile_2 + atk_ranged_2 + 
##     sp_atk_2 + speedster_2 + defence_2, family = binomial(link = "logit"), 
##     data = dp)
## 
## Deviance Residuals: 
##     Min       1Q   Median       3Q      Max  
## -1.7978  -1.0045   0.2275   1.1031   1.6541  
## 
## Coefficients: (1 not defined because of singularities)
##               Estimate Std. Error z value Pr(>|z|)  
## (Intercept)    2.70226    5.29804   0.510   0.6100  
## support        8.29985  528.47501   0.016   0.9875  
## versatile     -1.87832    1.51146  -1.243   0.2140  
## atk_ranged    -2.69874    1.88824  -1.429   0.1529  
## sp_atk        -0.77876    1.55093  -0.502   0.6156  
## speedster     -1.41344    1.47315  -0.959   0.3373  
## defence             NA         NA      NA       NA  
## support_2     -8.46513  528.47347  -0.016   0.9872  
## versatile_2    0.70974    0.38237   1.856   0.0634 .
## atk_ranged_2   1.29400    0.82956   1.560   0.1188  
## sp_atk_2       0.07928    0.29942   0.265   0.7912  
## speedster_2    0.46978    0.65556   0.717   0.4736  
## defence_2      0.03727    0.39650   0.094   0.9251  
## ---
## Signif. codes:  0 '***' 0.001 '**' 0.01 '*' 0.05 '.' 0.1 ' ' 1
## 
## (Dispersion parameter for binomial family taken to be 1)
## 
##     Null deviance: 138.63  on 99  degrees of freedom
## Residual deviance: 119.37  on 88  degrees of freedom
## AIC: 143.37
## 
## Number of Fisher Scoring iterations: 15
\end{verbatim}

Sadly results are nor meaningful neither interpretable, a different
approach should be used.

\hypertarget{some-spourious-regression---just-for-fun}{%
\subsubsection{Some spourious regression - just for
fun}\label{some-spourious-regression---just-for-fun}}

May be fun to regress the winning odds of each single Pokemon, despite
they won't be statistically significant.

\hypertarget{over-all-pokuxe9mon}{%
\paragraph{Over all Pokémon}\label{over-all-pokuxe9mon}}

\begin{Shaded}
\begin{Highlighting}[]
\NormalTok{glm.fit }\OtherTok{\textless{}{-}} \FunctionTok{glm}\NormalTok{(win }\SpecialCharTok{\textasciitilde{}}\NormalTok{ Absol }\SpecialCharTok{+}\NormalTok{ Aegislash }\SpecialCharTok{+}\NormalTok{ Azumarill }\SpecialCharTok{+}\NormalTok{ Blastoise }\SpecialCharTok{+}\NormalTok{ Blissey }\SpecialCharTok{+}\NormalTok{ Buzzwole }\SpecialCharTok{+}\NormalTok{ Charizard }\SpecialCharTok{+}\NormalTok{ Cinderace }\SpecialCharTok{+}\NormalTok{ Cramorant }\SpecialCharTok{+}\NormalTok{ Crustle }\SpecialCharTok{+}\NormalTok{ Decidueye }\SpecialCharTok{+}\NormalTok{ Delphox }\SpecialCharTok{+}\NormalTok{ Dodrio }\SpecialCharTok{+}\NormalTok{ Dragonite }\SpecialCharTok{+}\NormalTok{ Duraludon }\SpecialCharTok{+}\NormalTok{ Eldegoss }\SpecialCharTok{+}\NormalTok{ Espeon }\SpecialCharTok{+}\NormalTok{ Garchomp }\SpecialCharTok{+}\NormalTok{ Gardevoir }\SpecialCharTok{+}\NormalTok{ Gengar }\SpecialCharTok{+}\NormalTok{ Glaceon }\SpecialCharTok{+}\NormalTok{ Greedent }\SpecialCharTok{+}\NormalTok{ Greninja }\SpecialCharTok{+}\NormalTok{ Hoopa }\SpecialCharTok{+}\NormalTok{ Lucario }\SpecialCharTok{+}\NormalTok{ Machamp }\SpecialCharTok{+}\NormalTok{ Mamoswine }\SpecialCharTok{+}\NormalTok{ Mew }\SpecialCharTok{+}\NormalTok{ MrMime }\SpecialCharTok{+}\NormalTok{ Ninetales }\SpecialCharTok{+}\NormalTok{ Pikachu }\SpecialCharTok{+}\NormalTok{ Scizor }\SpecialCharTok{+}\NormalTok{ Slowbro }\SpecialCharTok{+}\NormalTok{ Snorlax }\SpecialCharTok{+}\NormalTok{ Sylveon }\SpecialCharTok{+}\NormalTok{ Talonflame }\SpecialCharTok{+}\NormalTok{ Trevenant }\SpecialCharTok{+}\NormalTok{ Tsareena }\SpecialCharTok{+}\NormalTok{ Tyranitar }\SpecialCharTok{+}\NormalTok{ Venusaur }\SpecialCharTok{+}\NormalTok{ Wigglytuff}
\NormalTok{               , }\AttributeTok{data =}\NormalTok{ dp, }\AttributeTok{family=}\FunctionTok{binomial}\NormalTok{(}\AttributeTok{link=}\StringTok{\textquotesingle{}logit\textquotesingle{}}\NormalTok{))}
\end{Highlighting}
\end{Shaded}

\begin{verbatim}
## Warning: glm.fit: fitted probabilities numerically 0 or 1 occurred
\end{verbatim}

\begin{Shaded}
\begin{Highlighting}[]
\FunctionTok{summary}\NormalTok{(glm.fit)}
\end{Highlighting}
\end{Shaded}

\begin{verbatim}
## 
## Call:
## glm(formula = win ~ Absol + Aegislash + Azumarill + Blastoise + 
##     Blissey + Buzzwole + Charizard + Cinderace + Cramorant + 
##     Crustle + Decidueye + Delphox + Dodrio + Dragonite + Duraludon + 
##     Eldegoss + Espeon + Garchomp + Gardevoir + Gengar + Glaceon + 
##     Greedent + Greninja + Hoopa + Lucario + Machamp + Mamoswine + 
##     Mew + MrMime + Ninetales + Pikachu + Scizor + Slowbro + Snorlax + 
##     Sylveon + Talonflame + Trevenant + Tsareena + Tyranitar + 
##     Venusaur + Wigglytuff, family = binomial(link = "logit"), 
##     data = dp)
## 
## Deviance Residuals: 
##      Min        1Q    Median        3Q       Max  
## -2.15765  -0.73826  -0.00007   0.65258   2.23185  
## 
## Coefficients:
##              Estimate Std. Error z value Pr(>|z|)  
## (Intercept)   -3.7186     6.6783  -0.557   0.5776  
## Absol         -0.4267     1.4904  -0.286   0.7747  
## Aegislash     -0.3059     1.7889  -0.171   0.8642  
## Azumarill      0.6711     2.0982   0.320   0.7491  
## Blastoise      0.4757     2.3384   0.203   0.8388  
## Blissey       -2.0624     2.3340  -0.884   0.3769  
## Buzzwole      -0.2171     1.5525  -0.140   0.8888  
## Charizard      1.9789     1.5265   1.296   0.1949  
## Cinderace      0.7130     1.8487   0.386   0.6997  
## Cramorant      3.9090     2.0634   1.894   0.0582 .
## Crustle        1.6400     1.8859   0.870   0.3845  
## Decidueye     -4.2462     2.0528  -2.068   0.0386 *
## Delphox       -0.2954     1.8229  -0.162   0.8713  
## Dodrio        -2.0708     1.9430  -1.066   0.2865  
## Dragonite     20.6186  2443.1429   0.008   0.9933  
## Duraludon      2.3831     2.6566   0.897   0.3697  
## Eldegoss     -13.3765  6522.6405  -0.002   0.9984  
## Espeon         1.2126     1.6183   0.749   0.4537  
## Garchomp      -4.0366     2.2587  -1.787   0.0739 .
## Gardevoir     -0.2491     1.4826  -0.168   0.8666  
## Gengar         0.5466     1.6378   0.334   0.7386  
## Glaceon        1.2529     1.9886   0.630   0.5287  
## Greedent      25.0845  2243.8556   0.011   0.9911  
## Greninja      -0.8560     1.5889  -0.539   0.5901  
## Hoopa        -35.3293  6965.1836  -0.005   0.9960  
## Lucario        3.8192     2.4756   1.543   0.1229  
## Machamp        3.0341     2.1586   1.406   0.1598  
## Mamoswine      4.0264     1.9002   2.119   0.0341 *
## Mew            1.5980     1.6265   0.982   0.3259  
## MrMime         2.6521     1.7042   1.556   0.1197  
## Ninetales      5.5103     3.0421   1.811   0.0701 .
## Pikachu       -0.3317     1.6932  -0.196   0.8447  
## Scizor         2.0729     2.1384   0.969   0.3324  
## Slowbro       -0.2147     2.3393  -0.092   0.9269  
## Snorlax        2.9405     1.7438   1.686   0.0918 .
## Sylveon        0.0113     1.9458   0.006   0.9954  
## Talonflame     1.7263     3.7487   0.461   0.6452  
## Trevenant     -0.9012     1.7182  -0.525   0.5999  
## Tsareena      -1.1144     1.7644  -0.632   0.5277  
## Tyranitar      2.2295     1.5617   1.428   0.1534  
## Venusaur      -2.8436     1.9805  -1.436   0.1511  
## Wigglytuff     1.7087     2.1694   0.788   0.4309  
## ---
## Signif. codes:  0 '***' 0.001 '**' 0.01 '*' 0.05 '.' 0.1 ' ' 1
## 
## (Dispersion parameter for binomial family taken to be 1)
## 
##     Null deviance: 138.629  on 99  degrees of freedom
## Residual deviance:  78.108  on 58  degrees of freedom
## AIC: 162.11
## 
## Number of Fisher Scoring iterations: 17
\end{verbatim}

\hypertarget{over-a-single-pokuxe9mon}{%
\paragraph{Over a single Pokémon}\label{over-a-single-pokuxe9mon}}

\begin{Shaded}
\begin{Highlighting}[]
\NormalTok{glm.fit }\OtherTok{\textless{}{-}} \FunctionTok{glm}\NormalTok{(win }\SpecialCharTok{\textasciitilde{}}\NormalTok{ Decidueye , }\AttributeTok{data =}\NormalTok{ dp, }\AttributeTok{family=}\FunctionTok{binomial}\NormalTok{(}\AttributeTok{link=}\StringTok{\textquotesingle{}logit\textquotesingle{}}\NormalTok{))}
\FunctionTok{summary}\NormalTok{(glm.fit)}
\end{Highlighting}
\end{Shaded}

\begin{verbatim}
## 
## Call:
## glm(formula = win ~ Decidueye, family = binomial(link = "logit"), 
##     data = dp)
## 
## Deviance Residuals: 
##     Min       1Q   Median       3Q      Max  
## -1.2450  -1.2450   0.2389   1.1113   1.8465  
## 
## Coefficients:
##             Estimate Std. Error z value Pr(>|z|)  
## (Intercept)   0.1576     0.2127   0.741   0.4586  
## Decidueye    -1.6617     0.8100  -2.051   0.0402 *
## ---
## Signif. codes:  0 '***' 0.001 '**' 0.01 '*' 0.05 '.' 0.1 ' ' 1
## 
## (Dispersion parameter for binomial family taken to be 1)
## 
##     Null deviance: 138.63  on 99  degrees of freedom
## Residual deviance: 133.26  on 98  degrees of freedom
## AIC: 137.26
## 
## Number of Fisher Scoring iterations: 3
\end{verbatim}

\hypertarget{a-balanced-team---decision-tree}{%
\subsubsection{A Balanced Team - Decision
Tree}\label{a-balanced-team---decision-tree}}

Trees are more powerful, they can take in account interaction between
variables and archive a sort of binary win-lose classification.

\begin{Shaded}
\begin{Highlighting}[]
\NormalTok{dp\_group }\OtherTok{=}\NormalTok{ dp[}\DecValTok{55}\SpecialCharTok{:}\DecValTok{60}\NormalTok{]}
\NormalTok{dp\_group}\SpecialCharTok{$}\NormalTok{win }\OtherTok{=}\NormalTok{ dp}\SpecialCharTok{$}\NormalTok{win}
\end{Highlighting}
\end{Shaded}

The accuracy of tree predictions is evaluated by computing the MSE. A
side dataset obtained by surveying some players is used as test set.

\begin{Shaded}
\begin{Highlighting}[]
\NormalTok{new\_data }\OtherTok{=}\FunctionTok{read.csv}\NormalTok{(}\StringTok{"datasets/survey.test.set.csv"}\NormalTok{)}
\end{Highlighting}
\end{Shaded}

\hypertarget{tree-0---tree-rbase}{%
\paragraph{Tree \#0 - tree rbase}\label{tree-0---tree-rbase}}

\begin{Shaded}
\begin{Highlighting}[]
\FunctionTok{library}\NormalTok{(tree)}
\end{Highlighting}
\end{Shaded}

\begin{verbatim}
## Warning: package 'tree' was built under R version 4.1.3
\end{verbatim}

\begin{Shaded}
\begin{Highlighting}[]
\NormalTok{tree }\OtherTok{\textless{}{-}} \FunctionTok{tree}\NormalTok{(win }\SpecialCharTok{\textasciitilde{}}\NormalTok{ ., }\AttributeTok{data =}\NormalTok{ dp\_group)}
\FunctionTok{summary}\NormalTok{(tree)}
\end{Highlighting}
\end{Shaded}

\begin{verbatim}
## 
## Regression tree:
## tree(formula = win ~ ., data = dp_group)
## Variables actually used in tree construction:
## [1] "support"    "atk_ranged" "speedster"  "versatile"  "sp_atk"    
## Number of terminal nodes:  9 
## Residual mean deviance:  0.2173 = 19.78 / 91 
## Distribution of residuals:
##     Min.  1st Qu.   Median     Mean  3rd Qu.     Max. 
## -0.85710 -0.33330  0.07143  0.00000  0.30430  0.80000
\end{verbatim}

\begin{Shaded}
\begin{Highlighting}[]
\FunctionTok{plot}\NormalTok{(tree)}
\FunctionTok{text}\NormalTok{(tree, }\AttributeTok{pretty =} \DecValTok{0}\NormalTok{)}
\end{Highlighting}
\end{Shaded}

\includegraphics{report_files/figure-latex/unnamed-chunk-34-1.pdf}

\begin{Shaded}
\begin{Highlighting}[]
\NormalTok{yhat }\OtherTok{\textless{}{-}} \FunctionTok{predict}\NormalTok{(tree, }\AttributeTok{newdata =}\NormalTok{ new\_data)}
\FunctionTok{plot}\NormalTok{(yhat, new\_data}\SpecialCharTok{$}\NormalTok{win)}
\FunctionTok{abline}\NormalTok{(}\DecValTok{0}\NormalTok{, }\DecValTok{1}\NormalTok{)}
\end{Highlighting}
\end{Shaded}

\includegraphics{report_files/figure-latex/unnamed-chunk-35-1.pdf}

\begin{Shaded}
\begin{Highlighting}[]
\FunctionTok{mean}\NormalTok{((yhat }\SpecialCharTok{{-}}\NormalTok{ new\_data}\SpecialCharTok{$}\NormalTok{win)}\SpecialCharTok{\^{}}\DecValTok{2}\NormalTok{)}
\end{Highlighting}
\end{Shaded}

\begin{verbatim}
## [1] 0.3752574
\end{verbatim}

\hypertarget{tree-1---rpart}{%
\paragraph{Tree \#1 - rpart}\label{tree-1---rpart}}

\begin{Shaded}
\begin{Highlighting}[]
\FunctionTok{library}\NormalTok{(}\StringTok{"rpart"}\NormalTok{)}
\FunctionTok{library}\NormalTok{(}\StringTok{"rpart.plot"}\NormalTok{)}
\NormalTok{tree1 }\OtherTok{\textless{}{-}} \FunctionTok{rpart}\NormalTok{(win }\SpecialCharTok{\textasciitilde{}}\NormalTok{ ., }\AttributeTok{data =}\NormalTok{ dp\_group, }\AttributeTok{control =} \FunctionTok{rpart.control}\NormalTok{(}\AttributeTok{cp =} \DecValTok{0}\NormalTok{, }\AttributeTok{minsplit =} \DecValTok{10}\NormalTok{, }\AttributeTok{maxsurrogate =} \DecValTok{10}\NormalTok{))}
\FunctionTok{printcp}\NormalTok{(tree1)}
\end{Highlighting}
\end{Shaded}

\begin{verbatim}
## 
## Regression tree:
## rpart(formula = win ~ ., data = dp_group, control = rpart.control(cp = 0, 
##     minsplit = 10, maxsurrogate = 10))
## 
## Variables actually used in tree construction:
## [1] atk_ranged defence    sp_atk     speedster  support    versatile 
## 
## Root node error: 25/100 = 0.25
## 
## n= 100 
## 
##            CP nsplit rel error xerror      xstd
## 1  5.2632e-02      0   1.00000 1.0263 0.0070528
## 2  3.4022e-02      1   0.94737 1.0288 0.0315830
## 3  1.4963e-02      4   0.84530 1.1582 0.0801283
## 4  1.2422e-02      6   0.81537 1.3196 0.1072443
## 5  1.1852e-02      7   0.80295 1.3520 0.1097617
## 6  8.6580e-03      8   0.79110 1.3641 0.1104505
## 7  7.7057e-03      9   0.78244 1.3553 0.1133937
## 8  6.8571e-03     10   0.77474 1.3550 0.1145089
## 9  4.9231e-03     11   0.76788 1.3667 0.1152427
## 10 4.6753e-03     12   0.76296 1.3681 0.1153486
## 11 1.1722e-03     13   0.75828 1.3795 0.1153362
## 12 7.6923e-05     14   0.75711 1.3846 0.1160244
## 13 0.0000e+00     15   0.75703 1.3846 0.1160244
\end{verbatim}

\begin{Shaded}
\begin{Highlighting}[]
\FunctionTok{rpart.plot}\NormalTok{(tree1)}
\end{Highlighting}
\end{Shaded}

\includegraphics{report_files/figure-latex/unnamed-chunk-36-1.pdf}

\begin{Shaded}
\begin{Highlighting}[]
\NormalTok{yhat }\OtherTok{\textless{}{-}} \FunctionTok{predict}\NormalTok{(tree1, }\AttributeTok{newdata =}\NormalTok{ new\_data)}
\FunctionTok{plot}\NormalTok{(yhat, new\_data}\SpecialCharTok{$}\NormalTok{win)}
\FunctionTok{abline}\NormalTok{(}\DecValTok{0}\NormalTok{, }\DecValTok{1}\NormalTok{)}
\end{Highlighting}
\end{Shaded}

\includegraphics{report_files/figure-latex/unnamed-chunk-37-1.pdf}

\begin{Shaded}
\begin{Highlighting}[]
\FunctionTok{mean}\NormalTok{((yhat }\SpecialCharTok{{-}}\NormalTok{ new\_data}\SpecialCharTok{$}\NormalTok{win)}\SpecialCharTok{\^{}}\DecValTok{2}\NormalTok{)}
\end{Highlighting}
\end{Shaded}

\begin{verbatim}
## [1] 0.4714233
\end{verbatim}

\hypertarget{tree-2---partykit}{%
\paragraph{Tree \#2 - partykit}\label{tree-2---partykit}}

\begin{Shaded}
\begin{Highlighting}[]
\FunctionTok{library}\NormalTok{(}\StringTok{"partykit"}\NormalTok{)}
\FunctionTok{library}\NormalTok{(}\StringTok{"party"}\NormalTok{)}
\NormalTok{tree2 }\OtherTok{\textless{}{-}} \FunctionTok{ctree}\NormalTok{(win }\SpecialCharTok{\textasciitilde{}}\NormalTok{ ., }\AttributeTok{data =}\NormalTok{ dp\_group, }\AttributeTok{controls =} \FunctionTok{ctree\_control}\NormalTok{(}\AttributeTok{testtype =} \StringTok{"Teststatistic"}\NormalTok{, }\AttributeTok{maxsurrogate =} \DecValTok{1}\NormalTok{, }\AttributeTok{mincriterion =} \FloatTok{0.5}\NormalTok{, }\AttributeTok{minsplit =} \DecValTok{1}\NormalTok{))}
\FunctionTok{plot}\NormalTok{(tree2, }\AttributeTok{cgp =} \FunctionTok{gpar}\NormalTok{(}\AttributeTok{fontsize =} \DecValTok{2}\NormalTok{))}
\end{Highlighting}
\end{Shaded}

\includegraphics{report_files/figure-latex/unnamed-chunk-38-1.pdf}

\begin{Shaded}
\begin{Highlighting}[]
\NormalTok{yhat }\OtherTok{\textless{}{-}} \FunctionTok{predict}\NormalTok{(tree2, }\AttributeTok{newdata =}\NormalTok{ new\_data)}
\FunctionTok{plot}\NormalTok{(yhat, new\_data}\SpecialCharTok{$}\NormalTok{win)}
\FunctionTok{abline}\NormalTok{(}\DecValTok{0}\NormalTok{, }\DecValTok{1}\NormalTok{)}
\end{Highlighting}
\end{Shaded}

\includegraphics{report_files/figure-latex/unnamed-chunk-39-1.pdf}

\begin{Shaded}
\begin{Highlighting}[]
\FunctionTok{mean}\NormalTok{((yhat }\SpecialCharTok{{-}}\NormalTok{ new\_data}\SpecialCharTok{$}\NormalTok{win)}\SpecialCharTok{\^{}}\DecValTok{2}\NormalTok{)}
\end{Highlighting}
\end{Shaded}

\begin{verbatim}
## [1] 0.3078129
\end{verbatim}

\hypertarget{consideration}{%
\paragraph{Consideration}\label{consideration}}

The best tree (\#2) archive a MSE of 0.30, and not by chances it's the
smallest tree. Smaller trees are less prone to overfit data, they return
better predictions and they are way easier to interpret.

\end{document}
